% Roll Number 56, Shruthin Vasanth

\textbf{\textcolor{LightMagenta}{Is regression a supervised learning technique? Justify your answer. Compare regression with classification with examples.\\ (Dec 2019, Q11 a)\hfill (Marks 5)}} \\[5pt]
\\
\textbf{\textcolor{purple}{Part 1:} }

   Linear Regression is Supervised because the data you have include both the input and the output. So, for instance, if you have a dataset for, say, car sales at a dealership. You have, for each car, the make, model, price, color, discount etc. but you also have the number of sales for each car. If this task was unsupervised, you would have a dataset that included, maybe, just the make, model, price, color etc. (not the actual number of sales) and the best you could do is cluster the data. Linear regression requires a quality check of the predicted output by comparing it with the actual desired output.

    

    \newline \\
\textbf{\textcolor{purple}{Part 2:} }
    
   \textbf{\textcolor{ReddishRose}{Regression} } is the process of finding a model or function for distinguishing the data into continuous real values instead of using classes or discrete values. It can also identify the distribution movement depending on the historical data. Because a regression predictive model predicts a quantity, therefore, the skill of the model must be reported as an error in those predictions.

    
    \textit{For Example:}  Predicting the amount of rainfall that we’ll get on a particular day based on historical data.
    
    \textbf{\textcolor{ReddishRose}{Classification}} is the process of finding or discovering a model or function which helps in separating the data into multiple categorical classes i.e. discrete values. In classification, data is categorized under different labels according to some parameters given in input and then the labels are predicted for the data. The derived mapping function could be demonstrated in the form of “IF-THEN” rules. The classification process deals with the problems where the data can be divided into binary or multiple discrete labels. \newline \\
 \textit{For Example:} Classifying whether a given image is a fruit or a vegetable.

