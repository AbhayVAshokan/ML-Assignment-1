% Roll Number 03, Ajay Krishnan P V

\textbf{\textcolor{LightMagenta}{Explain regression with an example. (May 2019 Q 11a) \hfill (Marks 5)}} \\[5pt]


Regression is a statistical method used in finance, investing, and other disciplines that attempts to determine the strength and character of the relationship between one dependent variable (usually denoted by Y) and a series of other variables (known as independent variables).\\
Regression helps investment and financial managers to value assets and understand the relationships between variables, such as commodity prices and stocks of business dealing in those commodities.\\
The two basic types of regression are simple linear regression and multiple linear regression, although there are non-linear regression methods for more complicated data and analysis. Simple linear regression uses one independent variable to explain or predict the outcome of the dependent variable Y, while multiple linear regression uses two or more independent variables to predict the outcome.Regression can help finance and investment professionals as well as professionals in other businesses. Regression can also help predict sales for a company based on weather, previous sales, GDP growth, or other types of conditions. The capital asset pricing model (CAPM) is an often-used regression model in finance for pricing assets and discovering costs of capital.\\

The general form of each type of regression is:\\
Simple linear regression: Y = a + bX + u\\
Multiple linear regression: Y = a + b1X1 + b2X2 + b3X3 + ... + btXt + u\\
Where:\\
Y = the variable that you are trying to predict (dependent variable).\\
X = the variable that you are using to predict Y (independent variable).\\
a = the intercept.\\
b = the slope.\\
u = the regression residual.\\

Regression takes a group of random variables, thought to be predicting Y, and tries to find a mathematical relationship between them. This relationship is typically in the form of a straight line (linear regression) that best approximates all the individual data points. In multiple regression, the separate variables are differentiated by using subscripts.\\

\textbf{\textcolor{purple}{A Real World Example of How Regression Analysis Is Used}}\\

Regression is often used to determine how many specific factors such as the price of a commodity, interest rates, particular industries, or sectors influence the price movement of an asset. The aforementioned CAPM is based on regression, and it is utilized to project the expected returns for stocks and to generate costs of capital. A stock's returns are regressed against the returns of a broader index, such as the S&P 500, to generate a beta for the particular stock.\\

Beta is the stock's risk in relation to the market or index and is reflected as the slope in the CAPM model. The return for the stock in question would be the dependent variable Y, while the independent variable X would be the market risk premium.\\

Additional variables such as the market capitalization of a stock, valuation ratios, and recent returns can be added to the CAPM model to get better estimates for returns. These additional factors are known as the Fama-French factors, named after the professors who developed the multiple linear regression model to better explain asset returns.\\

