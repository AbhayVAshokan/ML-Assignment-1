% roll no 44 - Mohammed Hasiq
\textbf{\textcolor{LightMagenta}{ Explain the basic elements of a Hidden Markov Model (HMM).Listany two applications of HMM. (Sep 2020) \hfill 4 marks}} \\[5pt]
{ An HMM can be completely described by the following elements:-
\newline(a) N is the number of the hidden states in the model. Individual states are denoted as S = {S1, S2, . . . , SN }, and the state at time t as qt.\newline
(b) M is the number of distinct observation symbols for each state and is called the alphabet size and correspond to the physical output of the system which is modelled.\newline
(c) The state transition probability distribution A = {aij}, where aij = P[qt = Sj |qt−1 = Si], 1 ≤ i, j ≤ N ,  aij ≥ 0 \newline
(d) The observation symbol probability distribution in state j, B = {bj (k)},where bj (k) = P[vk at t|qt = Sj ], 1 ≤ j ≤ N and 1 ≤ k ≤ M \newline
(e) The initial state distribution, describing the probability of beginning the state sequence in a certain initial state.
πi = P[q1 = Si], 1 ≤ i ≤ N \newline
(f) The observation sequence is denoted as O = O1O2 · · · OT .
λ represents the complete parameter set of a model, where λ = (A, B, π).\newline
HMM can be applied in Cryptanalysis, Speech Recognition, etc...}