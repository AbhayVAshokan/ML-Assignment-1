%Roll no 35 Jeevan George JOhn
\textbf{\textcolor{LightMagenta}{Describe the significance of Kernal functions in SVM. List any two kernel functions. (Sept 2020) \hfill 4 marks}} \\[5pt]
Kernel Function is a method used to take data as input and transform into the
required form of processing data.Kernel Function generally transforms the training set of
data so that a non-linear decision surface is able to transformed to a linear equation in a
higher number of dimension spaces. Basically, It returns the inner product between two
points in a standard feature dimension.
SVM algorithms use a set of mathematical functions that are defined as the kernel. The
function of kernel is to take data as input and transform it into the required form. Different
SVM algorithms use different types of kernel functions. These functions can be different
types. For example linear, nonlinear, polynomial, radial basis function (RBF), and sigmoid.
Introduce Kernel functions for sequence data, graphs, text, images, as well as vectors. The
most used type of kernel function is RBF. Because it has localized and finite response along
the entire x-axis.The kernel functions return the inner product between two points in a suit-
able feature space. Thus by defining a notion of similarity, with little computational cost
even in very high-dimensional spaces.\\
\textcolor{purple}{\underline{\smash{Gaussian Kernel}}:} It is used to perform transformation, when there is no prior knowledge about data. \\
\textit{from sklearn.svm import SVC \\
classifier = SVC(kernel=’rbf ’, random state = 0)\\
training set in x, y axis \\
classifier.fit(x train, y train) \\}

\textcolor{purple}{\underline{\smash{Polynomial Kernel}}:} It represents the similarity of vectors in training set of data in a feature space over polynomials of the original variables used in kernel.\\
\textit{from sklearn.svm import SVC \\
classifier = SVC(kernel =’poly’, degree = 4) \\
classifier.fit(x train, y train) \\}
