% Roll Number 48, Padmaraj Thambi

\textbf{\textcolor{LightMagenta}{Explain the concept of association rule analysis with its application. (Dec 2019, Q 13 (b)))\hfill (Mark 3)}} \\[5pt]
\text{Association Rule analysis, as the name suggests, association rules are simple.} {If/Then statements that help discover relationships between seemingly independent relational databases or other data repositories.

Most machine learning algorithms work with numeric datasets and hence tend to be mathematical. However, association rule analysis is suitable for non-numeric, categorical data and requires just a little bit more than simple counting.

Association rule analysis is a procedure which aims to observe frequently occurring patterns, correlations, or associations from datasets found in various kinds of databases such as relational databases, transactional databases, and other forms of repositories.

An association rule has 2 parts:

an antecedent (if) and
a consequent (then)
An antecedent is something that’s found in data, and a consequent is an item that is found in combination with the antecedent. Have a look at this rule for instance:

“If a customer buys bread, he’s 70\% likely of buying milk.”

In the above association rule, bread is the antecedent and milk is the consequent. Simply put, it can be understood as a retail store’s association rule to target their customers better. If the above rule is a result of a thorough analysis of some data sets, it can be used to not only improve customer service but also improve the company’s revenue.Association rules are created by thoroughly analyzing data and looking for frequent if/then patterns. Then, depending on the following two parameters, the important relationships are observed:
Support: Support indicates how frequently the if/then relationship appears in the database.
Confidence: Confidence tells about the number of times these relationships have been found to be true.
So, in a given transaction with multiple items, Association Rule analysis primarily tries to find the rules that govern how or why such products/items are often bought together. For example, peanut butter and jelly are frequently purchased together because a lot of people like to make PB&J sandwiches.}

\textbf{\textcolor{purple}{Application:}} \\
\textit{Market Basket Analysis:} 
This is the most typical example of association analysis. Data is collected using barcode scanners in most supermarkets. This database, known as the “market basket” database, consists of a large number of records on past transactions. A single record lists all the items bought by a customer in one sale. Knowing which groups are inclined towards which set of items gives these shops the freedom to adjust the store layout and the store catalog to place the optimally concerning one another.
This is the most typical example of association analysis. Data is collected using barcode scanners in most supermarkets. This database, known as the “market basket” database, consists of a large number of records on past transactions. A single record lists all the items bought by a customer in one sale. Knowing which groups are inclined towards which set of items gives these shops the freedom to adjust the store layout and the store catalog to place the optimally concerning one another.

Association Rule analysis has helped data scientists find out patterns they never knew existed.

