% roll no 43 - Misha Mohan
\textbf{\textcolor{purple}{  Define the following terms (a) sensitivity (b) Specificity (c) Preci-sion (d) Accuracy for a classification problem. (Dec 2018) \hfill 4 marks}} \\[5pt]
 \begin{enumerate}
     \item  \textcolor{purple} {Sensitivity} \\   
     Sensitivity is a measure of the proportion of actual positive cases that got predicted as positive or true positive. 
    Sensitivity = (True Positive)/(True Positive + False Negative)
    
    \item \textcolor{purple} {Specificity} \\ 
    It is defined as the proportion of actual negatives, which got predicted as the negative or true negative.
    Specificity = (True Negative)/(True Negative + False Positive)
    
    \item \textcolor{purple} {Precision} \\
    Precision is the number of correct positive results divided by the number of positive results predicted by the classifier.
    
    \item \textcolor{purple} {Accuracy} \\Accuracy is the ratio of number of correct predictions to the total number of input samples.
 \end{enumerate}
