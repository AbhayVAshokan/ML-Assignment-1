% Roll number 38, Jithin Rohith K R  
.
\textbf{\textcolor{LightMagenta}{Explain the steps involved in Expectation-Maximization algorithm?(May 2019) \hfill 4 marks}} \\[5pt]
\textcolor{purple}{Algorithm:}
\begin{itemize}
    \item Given a set of incomplete data, consider a set of starting parameters.
    \item Expectation step (E – step): Using the observed available data of the dataset,
estimate (guess) the values of the missing data.
    \item Maximization step (M – step): Complete data generated after the expectation (E)
step is used in order to update the parameters.
    \item Repeat step 2 and step 3 until convergence.
\end{itemize}
Initially, a set of initial values of the parameters are considered. A set of incomplete
observed data is given to the system with the assumption that the observed data comes
from a specific model.
\\
The next step is known as “Expectation” – step or E-step. In this step, we use the
observed data in order to estimate or guess the values of the missing or incomplete
data. It is basically used to update the variables.
\\
The next step is known as “Maximization”-step or M-step. In this step, we use the
complete data generated in the preceding “Expectation” – step in order to update the
values of the parameters. It is basically used to update the hypothesis.
\\
Now, in the fourth step, it is checked whether the values are converging or not, if yes,
then stop otherwise repeat step-2 and step-3 i.e. “Expectation” – step and
“Maximization” – step until the convergence occurs.
\\
